\documentclass[czech,11pt,a4paper]{article}
\usepackage[utf8]{inputenc}
\usepackage{a4wide}
\usepackage[pdftex,breaklinks=true,colorlinks=true,urlcolor=blue,
  pagecolor=black,linkcolor=black]{hyperref}
\usepackage[czech]{babel}

\pagestyle{empty}

\begin{document}

\begin{center}
  {\Large --- Posudek vedoucího bakalářské práce ---}
\end{center}

\vspace{.2cm}

\noindent \begin{tabular}{rp{.75\textwidth}}
  {\bf Bakalářská práce:} & Analýza a vizualizace srážkových dat z mikrovlnných telekomunikačních spojů pomocí GIS \\
  {\bf Student:} & Matěj Krejčí \\
  {\bf Vedoucí:} & Ing. Martin Landa, Ph.D. \\
  {\bf Oponent:} & Ing. Vojtěch Bareš, Ph.D. \\
\end{tabular}

\vspace{1cm}

Zadání bakalářské práce Matěje Krejčího vychází z grantu GAČR
(14-22978S) řešeného na katedře hydrauliky a hydrologie. Grant je
zaměřen na metody převodu útlumu mikrovlnných telekomunikačních spojů
(dále MV spojů) na srážkové intenzity a jejich následné využití
v~meteorologii a hydrologii.  \\

Hlavním úkolem studenta bylo v rámci své bakalářské práce přistoupit k
analýze a vizualizaci srážkových dat získaných z MV spojů v prostředí
geografických informačních systémů (GIS). Po domluvě s kolegy z
katedry hydrauliky a hydrologie byl zvolen jako implementační
framework prostředí open source GIS nástroje GRASS GIS.  \\

Student musel v poměrně krátkém časovém úseku načerpat základní
teoretické znalosti dané problematiky, seznámit se s prostředím
nástroje GRASS GIS především z pohledu programátora a vývojáře. Pro
vlastní implementaci nástrojů, které vznikly jako výstupy této práce,
byl použit programovací jazyk Python a příslušné knihovny systému
GRASS. Tyto nástroje byly publikovány jako doplňková součást systému
GRASS (tzv. Addons).  \\

Z pohledu vedoucího práce mohu konstatovat, že student plnil zadané
dílčí úkoly svědomitě. Mezivýsledky konzultoval nejen s vedoucím
práce, ale i se zadavateli z katedry hydrauliky a~hydrologie. Přesto
má výsledek jisté limity, na kterých by měl student do budoucna
zapracovat. Po formální stránce je text práce nepříliš šťastně
strukturován, vzhledem k tomu utrpěla čitelnost textu a nebyl dodržen
základní princip oddělení teoretické části, popisu metodických postupů
a výsledků práce. Z pohledu výstupu práce -- navržených nástrojů -- je
zde prostor na další vývoj či případné revize kódu (odstranění
závislosti na geodatabázi PostGIS, objektový návrh namísto
procedurálního a pod.).  \\

Za vedlejší přínos práce považuji posílení motivace studenta se dále
soustředit na vývoj GIS nástrojů. Výsledkem je jeho účast na
prestižním mezinárodním programu Google Summer of Code, což je
program společnosti Google na podporu studentů informatických oborů a
jejich zapojení do vývoje open source nástrojů.  \\

\newpage
Závěrem mohu konstatovat, že předložená bakalářská práce splňuje
všechny formální náležitosti a doporučuji ji k obhajobě. Bakalářskou
práci hodnotím stupněm

\vskip 2cm

\begin{center}
{\bf -- B (velmi dobře)  --}
\end{center}

\vskip 2cm

\begin{tabular}{lp{.25\textwidth}r}
& & \ldots\ldots\ldots\ldots\ldots\ldots\ldots \\
V~Solanech dne 20.6.2014 & & Ing. Martin Landa, Ph.D. \\
& & Fakulta stavební, ČVUT v Praze \\
\end{tabular}

\end{document}
